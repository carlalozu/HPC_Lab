\documentclass[unicode,11pt,a4paper,oneside,numbers=endperiod,openany]{scrartcl}

\usepackage{xcolor}
\usepackage{listings}
\usepackage{amsmath}

% Define custom verbatim environment with gray background
\lstnewenvironment{grayverbatim}{%
  \lstset{backgroundcolor=\color{gray!10}, % Adjust the shade of gray as desired
          frame=single,
          framerule=0pt,
          basicstyle=\ttfamily,
          breaklines=true,
          columns=fullflexible}
}{}

\lstnewenvironment{cppverbatim}{%
  \lstset{language=C++, % Set the language to C++
          backgroundcolor=\color{gray!10}, % Adjust the shade of gray as desired
          frame=single,
          framerule=0pt,
          basicstyle=\ttfamily,
          keywordstyle=\color{blue}, % Set the color for keywords
          commentstyle=\color{green!50!black}, % Set the color for comments
          stringstyle=\color{red}, % Set the color for strings
          breaklines=true,
          showstringspaces=false, % Don't show spaces within strings
          columns=fullflexible}
}{}

\input{assignment.sty}

\begin{document}


\setassignment
\setduedate{Monday 15 April 2024, 23:59 (midnight)}

\serieheader{High-Performance Computing Lab for CSE}{2024}
            {Student: CARLA JUDITH LOPEZ ZURITA}
            {Discussed with: FULL NAME}{Solution for Project 3}{}
\newline

\assignmentpolicy

In this report, we are solving Fisher's equation using the finite difference
method. The equation is given by:
\begin{align}
    \frac{\partial s}{\partial t} = D \nabla^2 u + R s (1 - s)
\end{align}
where $s$ is the concentration of a species, $D$ is the diffusion coefficient,
and $R$ is the reaction rate. The equation is solved on a 2D grid with
Dirichlet boundary conditions,
\begin{align}
    s(x, y, t) = 0.1 \quad \text{for} \quad x = 0, x = 1, y = 0, y = 1
\end{align}
and initial conditions is a circle of radius $1/8$ at the lower left quadrant the domain.
\section{Task: Implementing the linear algebra functions and the stencil
         operators [40 Points]}

The first part of the project is to implement the linear algebra functions and
the stencil operators. The linear algebra functions are implemented in the
\texttt{linalg.cpp} file. As first try, I implementd all the functions using
simple loops. I tried using iterators but since the Fields class is a coustom
class, I decided to put that idea on hold until I have a better understanding of
the code. On the other hand, the stencil operators are implemented in the
\texttt{operators.cpp}. Writing the interior grid points was very simple since
we already had the implementations for the boundary conditions.

Afterwards, I ran the code following the indications on the project description
and I got the following results in the terminal
\begin{grayverbatim}
================================================================================
                      Welcome to mini-stencil!
version   :: C++ Serial
mesh      :: 128 * 128 dx = 0.00787402
time      :: 100 time steps from 0 .. 0.005
iteration :: CG 300, Newton 50, tolerance 1e-06
================================================================================
--------------------------------------------------------------------------------
simulation took 0.203392 seconds
1513 conjugate gradient iterations, at rate of 7438.82 iters/second
300 newton iterations
--------------------------------------------------------------------------------
### 1, 128, 100, 1513, 300, 0.203392 ###
Goodbye!
\end{grayverbatim}
Which are the expected results, even if a bit slower than the results included
in the original document, but this is just one run and could be attributed to
the randomness of run-time of the processes. Additionally, I got the following content for
the \texttt{output.bin} file,
\begin{grayverbatim}
TIME: 0.005
DATA_FILE: output.bin
DATA_SIZE: 128 128 1
DATA_FORMAT: DOUBLE
VARIABLE: phi
DATA_ENDIAN: LITTLE
CENTERING: nodal
BRICK_ORIGIN: 0. 0. 0.
BRICK_SIZE: 1 1  1.0
\end{grayverbatim}
which produced Fig. \ref{im:final_time}.
\begin{figure}[h]
    \centering
    \includegraphics[width=0.5\textwidth]{../mini_app/output.png}
    \caption{The population concentration at time t = 0.005.}
    \label{im:final_time}
\end{figure}

\section{Task:  Adding OpenMP to the nonlinear PDE mini-app [60 Points]}
This section is dedicated to parallelizing the code using OpenMP. The first step
was to parallelize the linear algebra functions. I used the \texttt{omp
parallel for} directive to parallelize the loops. The stencil operators were


\end{document}
